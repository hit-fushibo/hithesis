% !Mode:: "TeX:UTF-8"

\hitsetup{
  %******************************
  % 注意:
  %   1. 配置里面不要出现空行
  %   2. 不需要的配置信息可以删除
  %******************************
  %
  %=====
  % 秘级
  %=====
  statesecrets={公开},
  natclassifiedindex={TM301.2},
  intclassifiedindex={62-5},
  %
  %=========
  % 中文信息
  %=========
  ctitleone={基于金字塔结构的迭代优},%本科生封面使用
  ctitletwo={化医学图像配准方法研究},%本科生封面使用
  ctitlecover={基于金字塔结构的迭代优化医学图像配准方法研究},%放在封面中使用,自由断行
  ctitle={基于金字塔结构的迭代优化医学图像配准方法研究},%放在原创性声明中使用
  % csubtitle={一条副标题}, %一般情况没有,可以注释掉
  cxueke={工学},
  csubject={计算机科学与技术},
  caffil={计算学部},
  cauthor={符世博},
  csupervisor={骆功宁副教授},
  cassosupervisor={某某某教授}, % 副指导老师
  ccosupervisor={某某某教授}, % 联合指导老师
  % 如果是第一封面的日期要手动设置,需要取消注释下一行,并将内容改为“规范”中要求的封面第一页最下方的日期
  firstpagecdate={2025年6月},
  % 日期自动使用当前时间,若需指定按如下方式修改:
  cdate={2025年6月},
  cstudentid={2021113140},
  cstudenttype={学术学位论文}, %非全日制教育申请学位者
  cnumber={no9527}, %编号
  cpositionname={哈铁西站}, %博士后站名称
  cfinishdate={20XX年X月---20XX年X月}, %到站日期
  csubmitdate={20XX年X月}, %出站日期
  cstartdate={3050年9月10日}, %到站日期
  cenddate={3090年10月10日}, %出站日期
  %(同等学力人员)、(工程硕士)、(工商管理硕士)、
  %(高级管理人员工商管理硕士)、(公共管理硕士)、(中职教师)、(高校教师)等
  %
  %
  %=========
  % 英文信息
  %=========
  etitle={A Pyramid-Based Iterative Optimization Method for Medical Image Registration},
  esubtitle={This is the sub title},
  exueke={Engineering},
  esubject={Computer Science and Technology},
  eaffil={\emultiline[t]{School of Mechatronics Engineering \\ Mechatronics Engineering}},
  eauthor={Yu Dongmei},
  esupervisor={Professor XXX},
  eassosupervisor={XXX},
  % 日期自动生成,若需指定按如下方式修改:
  edate={December, 2017},
  estudenttype={Master of Art},
  %
  % 关键词用“英文逗号”分割
  ckeywords={医学图像配准,互学习,无监督学习,金字塔网络},
  ekeywords={medical image registration,Mutual learning,unsupervised learning,pyramid network},
}

\begin{cabstract}

  医学图像配准作为医学影像分析的核心技术,在疾病诊断与治疗规划中具有重要价值。针对现有无监督配准方法存在的对训练数据的依赖和对空间特征变化敏感的问题,本文基于互学习范式的医学图像配准框架MutualReg,通过递归训练融合基于金字塔和自注意力机制的PAN网络与基于CNN架构的RegCST网络。该框架通过教师网络(PAN)与学生网络(RegCST)的递归式双向知识蒸馏,结合VRC模块的动态掩膜机制,实现了无监督配准性能的系统性提升。研究首先构建了面向大规模脑部MRI数据的预处理流程,通过基于模态无关邻域描述符(MIND)的相似性筛选策略生成1835对高质量训练样本;继而改进PAN网络的损失函数,引入梯度扩散正则化替代传统Bending Energy约束,使其在LPBA40验证集上的Dice系数提升至72.2\%;针对RegCST网络的显存瓶颈设计分块循环训练策略,结合多任务监督框架将配准精度提高至70.6\%。通过引入VRC模块,本研究实现了知识蒸馏过程中可靠体素位置的动态筛选,三轮递归互学习后模型Dice系数达72.5\%,较基线方法提升2.3个百分点,形变场平滑性指标SDlogJ优化至0.139。实验表明,所提出的互学习框架通过跨网络特征融合与误差修正机制,显著提升了复杂解剖结构的对齐精度,为多模态医学图像配准提供了新的解决方案。
\end{cabstract}

\begin{eabstract}
  Medical image registration, as a core technology in medical image analysis, holds significant value for disease diagnosis and treatment planning. To address the dependency on training data and sensitivity to spatial feature variations in existing unsupervised registration methods, this study proposes MutualReg, a mutual learning paradigm-based medical image registration framework. This framework recursively integrates a pyramid and self-attention mechanism-enhanced PAN network with a CNN-based RegCST network through bidirectional knowledge distillation. By implementing recursive mutual learning between the teacher network (PAN) and student network (RegCST), combined with the dynamic masking mechanism of the VRC module, systematic performance improvement in unsupervised registration is achieved. The research first establishes a preprocessing pipeline for large-scale brain MRI data, generating 1,835 high-quality training pairs through a modality-independent neighborhood descriptor (MIND)-based similarity screening strategy. Subsequent improvements to PAN's loss function by replacing traditional Bending Energy constraints with gradient diffusion regularization elevate the Dice coefficient to 72.2\% on the LPBA40 validation set. For RegCST's memory limitations, a patch-based cyclic training strategy is designed, combined with multi-task supervision to enhance registration accuracy to 70.6\%. Through the VRC module, dynamic screening of reliable voxel locations during knowledge distillation is realized. After three recursive mutual learning cycles, the model achieves 72.5\% Dice coefficient, outperforming baseline methods by 2.3 percentage points, with the deformation field smoothness metric SDlogJ optimized to 0.139. Experiments demonstrate that the proposed mutual learning framework significantly improves alignment accuracy for complex anatomical structures through cross-network feature fusion and error correction mechanisms, providing a novel solution for multimodal medical image registration.
\end{eabstract}
