% !Mode:: "TeX:UTF-8" 
\begin{conclusions}



本研究针对无监督医学图像配准中面对不同图像域中空间特征变化时对分布外数据配准性能下降的问题,基于互学习范式的配准优化框架,提出了一种结合金字塔结构和自注意力机制的解决思路,取得以下创新性成果:

\begin{enumerate}
    \item 提出了基于互学习配准框架融合金字塔结构与局部注意力机制的解决思路。通过构建异构教师-学生网络体系,将PAN网络的多尺度特征提取能力与RegCST网络的局部优化优势相结合,建立了双向知识蒸馏的递归训练机制,在LPBA40数据集上实现了72.5\%的Dice系数,较单网络训练提升2.8\%。
    \item 验证了VRC模块动态监督机制的有效性。通过消融实验对比有无VRC模块的配准性能,对于一轮递归训练,使用VRC模块Dice系数相较于未使用提升0.8个百分点,同时SDlogJ指标波动幅度控制在4.3\%以内。
    \item 开发了面向大规模数据的自适应训练策略。针对RegCST网络的显存限制,提出分块循环训练方法,峰值显存占用降低56.7\%;改进的混合损失函数通过动态权重调整机制,使模型收敛速度提升32.1\%,Dice系数在10个训练周期内达到67.3\%。
    \item 构建了多维度评估验证体系。通过解剖结构可视化对齐、差分图像分析与定量指标(Dice、SDlogJ)的综合评估,证实了所提方法的配准优势。
\end{enumerate}

未来工作将在以下方面展开深入研究:

\begin{enumerate}
    \item 拓展互学习框架至多模态配准场景,探索跨模态特征对齐机制。
    \item 研究基于互学习的递归终止策略,优化训练效率。
    \item 融合拓扑保持约束与生物力学先验知识,进一步提升形变场的生理合理性。
\end{enumerate}

本研究为智能医学影像分析提供了新的理论方法,具有重要的临床应用价值。

\end{conclusions}
